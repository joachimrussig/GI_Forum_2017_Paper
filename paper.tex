% !TeX program = pdflatex
% !TeX encoding = utf8
% !TeX spellcheck = en_GB
% !BIB program = biber

\documentclass[a4paper]{scrartcl}
\usepackage[utf8]{inputenc}
\usepackage[english]{babel}
\usepackage{amsmath,amssymb,amsfonts}
\usepackage{siunitx}
\usepackage{hyperref}

\usepackage{csquotes}
\usepackage[style=authoryear,backend=biber]{biblatex} 
\addbibresource{literature.bib}

\DeclareCiteCommand{\citeyear}
{}
{\bibhyperref{\printfield{year}}}
{\multicitedelim}
{}

\DeclareCiteCommand{\citeyear*}
{}
{\bibhyperref{\printfield{year}\printfield{extrayear}}}
{\multicitedelim}
{}

%opening
\title{Decision Support for Route Planning to Reduce Heat Stress Considering the Time of the Day}
\author{}

\begin{document}

\maketitle

\begin{abstract}
	Heat stress is a serious risk, in particular for certain groups like elderly or patients with multiple sclerosis or heart disease. Developments like the ageing of society, the increasing urbanisation (urban heat island effect) and the climate change are increasing the risk that people are affected by heat stress. One way to reduce those risks is to adapt the everyday behaviour, e.g. by performing purchases in the supermarket or pharmacy in the morning or evening when temperatures are lower. 
	
	Therefore we are presenting two different approaches for decision support tools that can help people to adapt their everyday behaviour. At first we're presenting a route planer for pedestrians that can find a route with minimal heat exposure. The second approach we're proposing is a tool that supports the user to select the point in time with a minimal risk of heat stress, considering e.g. the opening hours of a shop. In both cases we are utilizing, among other, remote sensing data of a thermal flight scanner. 
	
	Our results are showing that both approaches are able to reduce the heat exposure and therefore can help people to decrease the risk of heat stress in their everyday life.
\end{abstract}

% !TeX encoding = utf8
% !TeX spellcheck = en_GB

\section{Introduction}

Heat is an important factor to human health and comfort. High temperatures cannot only lead to a discomfort, it also has serious negative effects on the health as well as the ability to work. 

In numerous studies an increase in both mortality and
morbidity has been associated with a high ambient temperature \parencite{Basu2009}. The most well-known example in recent history is the 2003 heat wave in Europe. 
Certain groups are especially vulnerable to heat stress such as older people or people with health problems \parencite{Huebler2007}. For patients with multiple sclerosis an increased body temperature can lead to a worsening of their symptoms \parencite{Davis2010}.

Developments like the ageing of society, the increasing urbanisation and the climate change is making the adaptation to heat stress danger more and more important. Due to the tendency that a rising number of people is moving into the cities, the urban heat island effect (UHI) is gaining more importance in the future. The UHI effect states that an urban area is significantly warmer than surrounding rural areas \parencite{Prashad2014}. 

Individuals can reduce their heat stress by adapting their everyday behaviour. In a city, most typical activities are in walking distance. These can range from going to a grocery store to the visit of a doctor. While these activities cannot be omitted, it is possible to use different routes or change the time when they are conducted. In doing so, one can easily reduce the heat stress without negative impact on the quality of life. 

In this paper we use this reasoning into a two-step approach to help individuals reduce their heat stress. We apply a routing algorithm to compute the optimal path in regard to the heat stress. This algorithm is then used to determine the optimal point in time to conduct typical everyday activities. 
  
\subsection{Related Work} 

\subsubsection{Heat Stress}
The impact of heat on the human body has long been a subject of study. Thermal comfort plays a key role, which describes climatic conditions consider comfortable. 

\textcite{Staiger1997} state that only a complete heat budget model of the human body is sufficient to make any reliable statements regarding the influence of heat on the body. Some well-known indices that consider a complete human heat budget model are for instance:
\begin{inparaenum}[(1)]
  \item \citeauthor{Steadman1979}'s heat index (\citeauthor{Steadman1979} \citeyear*{Steadman1979}, \citeyear*{Steadman1979a}),
  \item the predicted mean vote (PMV) \parencite{Fanger1973},
  \item the perceived temperature \parencite{Staiger1997,Jendritzky2000},
  \item and the universal thermal climate Index UTCI \parencite{Jendritzky2010}.
\end{inparaenum}

For all these indices the following meteorological parameters are important:
\begin{inparaenum}[(1)]
\item air temperature,
\item water vapour pressure,
\item wind velocity 
\item and mean radiant temperature \parencite{Jendritzky2010}.
\end{inparaenum}

Based on the availability of data, in this paper we will use \citeauthor{Steadman1979}'s heat index \parencite{Steadman1979} and, as a simple comparison measure, the air temperature.

\subsubsection{Health Optimal Pedestrian Routing}
Several research projects have considered environmental factors for pedestrian routing in the past, with the goal to find routes which are healthier. For instance, \textcite{Sharker2012} are proposing a method to find a health optimal route, considering several environmental factors like complexity of the walking trail and weather. A method to find a route with a minimal pollution exposure has been proposed by \textcite{Hasenfratz2015}.

The NaviComf framework for pedestrian routing proposed by \textcite{Dang2013} improves the comfort considering environmental factors varying over time. Their  framework uses a multi-factor cost model for the evaluation of the route and enables a consideration of heterogeneous environmental information from multi-modal sensors. To find an optimal route \textcite{Dang2013} are proposing three different algorithms, a bounded depth-first search algorithm, an adjustable dynamic planning algorithm and a heuristic particle planning algorithm. As a sample application, the authors implemented a routing application for thermal comfort navigation. The meteorological data used for this sample application have been collected using a network of 40 micro-climate sensor nodes which detected air temperature and relative humidity. 

In contrast to the existing work, we contribute an approach which does not rely on extensive sensor networks. We achieve this by combining remote sensing data with fixed weather stations and the use of a static routing algorithm.

% \input{basics}

\section{Story}
\begin{itemize}
\item Motivation
\begin{itemize}
\item heat is critical for humans
\item particular problems for risk groups (illness, old, etc)
\item walking to areas often done action in particular to typical locations (shopping, health care, etc)
\item those actions have to be done   
\end{itemize}
\item Problem definition
\begin{itemize}
\item how to minimize the impact of heat stress on walking paths?
\item subdivided into:
\begin{itemize}
\item for a given route?
\item When and how to walk typical routes, e.g. pharmacy, doctor
\end{itemize} 
\end{itemize}
\item method:
\begin{itemize}
\item Goal of this paper: We want to provide a routing method to minimize the heat stress of typical walking actions
\item We do so by providing a value function for heat stress which can be used in route planning
\end{itemize}
\item Evaluation:
\begin{itemize}
\item Our data set
\item Evaluation 1 on routes given date in time.  (pure routing approach) and its metrics
\item Evaluation 2: producing queries for typical walking tasks (our selection) and their evaluation
\item implementation as a demonstrator is produced and available at $homepage$
\end{itemize}
\item Conclusion and contribution
\end{itemize}


\printbibliography

\end{document}
