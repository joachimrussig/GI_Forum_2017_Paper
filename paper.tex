% !TeX program = pdflatex
% !TeX encoding = utf8
% !TeX spellcheck = en_GB
% !BIB program = biber

\documentclass[a4paper,parskip=full]{scrartcl}
\usepackage[T1]{fontenc}
\usepackage[utf8]{inputenc}
\usepackage[english]{babel}
\usepackage{amsmath,amssymb,amsfonts}
\usepackage{siunitx}
\usepackage{hyperref}

\usepackage{csquotes}
\usepackage[style=authoryear,backend=biber]{biblatex} 
\addbibresource{literature.bib}

\DeclareCiteCommand{\citeyear}
{}
{\bibhyperref{\printfield{year}}}
{\multicitedelim}
{}

\DeclareCiteCommand{\citeyear*}
{}
{\bibhyperref{\printfield{year}\printfield{extrayear}}}
{\multicitedelim}
{}

%opening
\title{Decision Support for Route Planning to Reduce Heat Stress Considering the Time of the Day}
\author{}

\begin{document}

\maketitle

\begin{abstract}
	Heat stress is a serious risk, in particular for certain groups like elderly or patients with multiple sclerosis or heart disease. Developments like the ageing of society, the increasing urbanisation (urban heat island effect) and the climate change are increasing the risk that people are affected by heat stress. One way to reduce those risks is to adapt the everyday behaviour, e.g. by performing purchases in the supermarket or pharmacy in the morning or evening when temperatures are lower. 
	
	Therefore we are presenting two different approaches for decision support tools that can help people to adapt their everyday behaviour. At first we're presenting a route planer for pedestrians that can find a route with minimal heat exposure. The second approach we're proposing is a tool that supports the user to select the point in time with a minimal risk of heat stress, considering e.g. the opening hours of a shop. In both cases we are utilizing, among other, remote sensing data of a thermal flight scanner. 
	
	Our results are showing that both approaches are able to reduce the heat exposure and therefore can help people to decrease the risk of heat stress in their everyday life.
\end{abstract}

% !TeX encoding = utf8
% !TeX spellcheck = en_GB

\section{Introduction}

Heat is an important factor to human health and comfort. High temperatures cannot only lead to a discomfort, such as sweating, it also has serious negative effects on the health as well as on the ability to work. 

In numerous studies an increase in both mortality and
morbidity has been associated with a high ambient temperature \parencite{Basu2009}. The most well-known example in recent history is the 2003 heat wave in Europe. 
Certain groups are especially vulnerable to heat stress such as older people or people with health problems \parencite{Huebler2007}. For patients with multiple sclerosis an increased body temperature can lead to a worsening of their symptoms \parencite{Davis2010}.

Developments like the ageing of society, the increasing urbanisation and the climate change is making the adaptation to heat stress danger more and more important. Due to the tendency that a rising number of people is moving into the cities, the urban heat island effect (UHI) is gaining more importance in the future. The UHI effect states that an urban area can be  \SIrange{8}{12}{\celsius} warmer than the rural areas \parencite{Prashad2014}. 

Individuals can reduce their heat stress by adapting their everyday behaviour. In a city, most typical activities are in walking distance. These can range from going to a grocery store to the visit of a doctor. And while these activities cannot be omitted, it is possible to use different routes or change the time when they are conducted. In doing so, one can easily reduce the heat stress without any negative impact on the quality of life. 

In this paper, we use this reasoning into a two-step approach to help individuals reduce their heat stress. First, we apply a routing algorithm to compute the optimal path in regard to the heat stress. This algorithm is then used to determine the optimal point in time to conduct typical everyday activities. 
  
\subsection{Related Work} 

\subsubsection{Heat Stress}
The impact of heat on the human body has long been a subject of study. Thermal comfort plays a key role, which describes climatic conditions consider comfortable. 

\textcite{Staiger1997} state that only a complete heat budget model of the human body is sufficient to make any reliable statements regarding the influence of heat on the body. Some well-known indices that consider a complete human heat budget model are for instance:
\begin{inparaenum}[(1)]
  \item \citeauthor{Steadman1979}'s heat index (\citeauthor{Steadman1979} \citeyear*{Steadman1979}, \citeyear*{Steadman1979a}),
  \item the predicted mean vote (PMV) \parencite{Fanger1973},
  \item the perceived temperature \parencite{Staiger1997,Jendritzky2000},
  \item and the universal thermal climate Index UTCI \parencite{Jendritzky2010}.
\end{inparaenum}

For all these indices the following meteorological parameters are important:
\begin{inparaenum}[(1)]
\item air temperature,
\item water vapour pressure,
\item wind velocity 
\item and mean radiant temperature \parencite{Jendritzky2010}.
\end{inparaenum}

Based on the availability of data, in this paper we will use \citeauthor{Steadman1979}'s heat index \parencite{Steadman1979} and, as a simple comparison measure, the air temperature.

\subsubsection{Health Optimal Pedestrian Routing}
Several research projects have considered environmental factors for pedestrian routing in the paste, with the goal to find routes which are more healthy. For instance, \textcite{Sharker2012} are proposing a method to find a health optimal route, considering several environmental factors like complexity of the walking trail (slope etc.) and weather (only “Good”, “Fair” or “Bad”). A method to find a route with a minimal pollution exposure has been proposed by \textcite{Hasenfratz2015}.

The NaviComf framework for pedestrian routing proposed by \textcite{Dang2013} enables to improve the comfort, considering environmental factors varying over time. The proposed framework uses a multi-factor cost model for the evaluation of the route and enables them to consider heterogeneous environmental information from multi-modal sensors, like air temperature and humidity. To find a optimal route \textcite{Dang2013} are proposing three different algorithms,  a bounded depth-first search algorithm, an adjustable dynamic planning algorithm and a heuristic particle planning algorithm. As a sample application, the authors implemented a routing app for thermal comfort navigation. The meteorological data used for this sample application have been collected using a network of 40 micro-climate sensor nodes which detected air temperature and relative humidity. 

In contrast to the existing work, we contribute an approach which does not rely on extensive sensor networks. We achieve this by combining remote sensing data with in-situ weather stations and the use of a static routing algorithm.
% !TeX encoding = utf8
% !TeX spellcheck = en_GB

\section{Minimize Heat Exposure}

We are presenting to possible ways how people can be supported to reduce their heat stress risk in their everyday life. First we're presenting an approach to find a route for pedestrian with a minimal heat exposure. On this basis, we show an approach to find a point in time with a minimal heat exposure, for instance to go shopping in a supermarket.

\subsection{Finding a Route with Minimal Heat Exposure}

Finding a route with minimal heat exposure can either be modelled as time-dependent or as a time-expanded routing problem. In case of a time-dependent routing problem the edge weighting function is not static and may vary over time. That means that many speed up techniques developed for static routing problems like bi-directional search cannot simply be applied \parencite{Delling2009}. In the case of time-expanded routing problem every node is replaced by multiple nodes representing the node at a different time. The advantage is, that finding the shortest path in the expended graph is a static routing problem, but there can be a huge blow up of the number of nodes in the network. Those the time-expanded modelling is usually applied on e.g. timetable information \parencite{Delling2009} and we are focusing on the time-depended approach. 

\subsubsection{Modelling as a Time-Dependent Routing Problem}

To find a optimal route we have to model the road network in an appropriate manner. In the following, we are representing the road network as undirected graph $G=(V,E,w_d,w_h)$, where $V$ is the set of vertices or nodes (e.g. junctions) and $E\subseteq V\times V$ is the set of edges (e.g. road segments) each connecting a pair of nodes. Furthermore $w_d: E \to \mathbb{R}_{\geq 0}$ and $w_h: E \times T \to \mathbb{R}_{\geq 0}$ are to edge weighting function, at which:
\begin{itemize}
	\item $w_d(e)$ is the length of the edge $e$, and
	\item $w_h(e, t)$ is the heat exposure of edge $e$ at time $t$.
\end{itemize}   
Below, a path $p$ from node $v_0$ to node $v_k$ starting a time $t_0$ is denoted as sequence of edge time pairs $((e_{v_0v_1},t_0),(e_{v_1v_2},t_1),\dots, (e_{v_{k-1}v_k},t_{k-1}))$, where $t_i$ is the time at which node $v_i$ is leaved. The weight of an edge is fixed at the time the traversing of the edge is started \parencite[the so-called frozen link model,][]{Orda1990}. The point in time $t_i$ can be computed as follows: $t_i := t_{i-1} + t_{walk}(e_{v_{i-1},v_i})$ where $t_{walk}(e_{v_{i-1},v_i})$ is the time needed by a pedestrian to traverse the edge $e_{v_{i-1},v_i}$. The starting time $t_0$ is ether given or set to $0$. 

To compute the weight of a path $w_h(p)$ the following formula can be applied:
	\begin{equation}\label{eq:path-weight}
		w_h(p) := \sum_{(t,e) \in p} w_h(e, t).
	\end{equation}
Those means we are looking for the path $p^*$ from a node $u$ to a node $v$ that has the minimal weight of all possible path from $v$ to $u$. Below, we are using $w_h(p, t)$ to denote the weight of the path $p$ starting at time $t$. 

The time-dependent routing problem is $\mathcal{NP}$-hard, if it is not allowed to wait on a node and the FIFO (first in, first out) or non-passing property is not fulfilled \parencite{Orda1990}. A edge weighting function $w: E \times T \to \mathbb{R}_{\geq 0}$ stratifies the FIFO property if for all edges $e=(u,v)\in E$ and all points in time $t, t' \in T$ with $t \leq t'$ the following holds \parencite{Ahn1991}:
   \begin{equation}\label{eq:fifo-property}
  	 t + w(e,t) \leq w(e, t').
   \end{equation}
In other words, a weighting function $w$ fulfils the FIFO property when the numerator (change of the edge weight)
decreases not faster than the denominator (change in actual time) increases, i.e. the slope of the weighting function is greater or equals to $-1$ \parencite{Kaufman1993}.  

Usually, we cannot assume that $w_h$ fulfils the FIFO property, because the function most of the time depends on the air temperature and the decrease of the air temperature can be more than $-1$. Since, most people are not willing to wait at a node as well, finding a route with a minimal heat exposure is $\mathcal{NP}$-hard. Therefore, 
hereafter the edge weighting is frozen at the starting time $t_0$ so that we have static route planning problem a classic algorithms like  \citeauthor{Dijkstra1959}'s algorithm \parencite{Dijkstra1959} can be applied. 

\subsubsection{The Edge Weighting Function}

\begin{itemize}
	\item general modelling (distance times heat exposure), $T^{comfort}$ as lower bound
	\item mapping raster cells to road network
	\item explain data available (or should that be part of )
\end{itemize}
 
 \subsection{Finding the Optimal Point in Time}
 
 \begin{itemize}
 	\item general idea
 	\item the three steps to find the optimal point in time
 		\begin{enumerate}
 			\item Nearby search
 			\item Find the optimal point in time
 			\item ranking by the optimal value found in step 2
 		\end{enumerate}
 \end{itemize}
 
 \subsubsection{Modelling as a Optimization Problem}
 
 \begin{itemize}
 	\item modelling of the constrains
 	\item objective function (only the modelling where the heat exposure of the path is used as objective function, equation 4.5 in the master thesis)
 \end{itemize}

 \subsubsection{Optimization}
 
 \begin{itemize}
 	\item transformation of the constrains to a lower and a upper bound
 	\item introduction of a penalty term if the constrains are violated
 	\item simple optimization algorithms without derivatives like Brent's method can be applied   
 \end{itemize}
% !TeX encoding = utf8
% !TeX spellcheck = en_GB

\section{Evaluation}

\subsection{Data}
\label{sec:data-sets}
As map data, we use the data from the OpenStreetMap (OSM) project \parencite{OSMF2016}. 
For weather data, we use the hourly air temperature and relative humidity values originating from the weather station of the German Weather Service (Deutscher Wetterdienste, DWD) in Rheinstetten near Karlsruhe \parencite{DWD2016}. For a finer spatial resolution we use remote sensing data of a thermal scanner flight provided by the Nachbarschaftsverband Karlsruhe (NVK). The data set consists of two scans, recorded in the morning and in the evening of the 26 September 2008. The data is covering an area of  $\SI{25 805}{\meter} \times \SI{39 555}{\meter}$ (EW NS) and have a resolution of $5\,161 \times 7\,911$  pixels. The measured surface temperature is in the range of \SIrange{-1.7}{18.3}{\celsius} (morning and evening). The average surface temperature of the data sets cropped to the evaluated area is \SI{4.18}{\celsius} (morning) respectively \SI{11.24}{\celsius} (evening).  


\subsection{Data Preparation}
The OSM data set is cropped to the evaluated area. Afterwards, all ways tagged with \verb|highway|, \verb|railway=platform| or \verb|public_transport=platform| are extracted to obtain the road network.

To compute the edge weights as described above in section \ref{sec:edge-weighting} we need to make some assumptions. That is because the weather data that we use lack either an appropriate spatial resolution or the required temporal resolution. We therefore assume that the actual spatial variation of the temperature conforms with the spatial variation of the thermal scans (deviation from the mean value). For the relative humidity, we assume a constant value over the study area. For the temporal variation, we assume that the temporal variation in the examined area corresponds to the temporal variation measure at the weather station. We apply the morning scan to timestamps between  00:00 and 11:59 and the evening scan between 12:00 and 23:59.

We compute the air temperature at time $t\in T$ for the raster cell $c_{ij}$ as follows:
\begin{equation}
\label{eq:derived-temperature}
T_a(t, c_{ij}) = \begin{cases}
T_{a}^{station}(t) + \delta_{ij}^{morning} & \text{if $0 \leq t < 12$,}\\
T_{a}^{station}(t) + \delta_{ij}^{evening} & \text{if $12 \leq t < 24$,}
\end{cases}
\end{equation}
where $T_{a}^{station}(t)$ is the air temperature measured at the weather station at time $t$ and $\delta^{morning}_{ij}$ respectively  $\delta^{evening}_{ij}$ is the deviation of the raster cell $c_{ij}$ from the mean of all raster cells from the morning respectively evening scan.

We compute an approximation of Steadman's heat index as proposed by \textcite[77]{Stull2011}. Since the heat index is only defined for an air temperature between \SI{20}{\celsius} and \SI{50}{\celsius}, the air temperature is used as a fall-back value. If the air temperature drops in a raster cell below a comfort threshold $T_a^{comfort}$ or $T_{HI}^{comfort}$, that comfort value will be used, because temperature below this threshold are not considered harmful. 

For the implementation of the routing the GraphHopper framework for Java \parencite{GraphHopper2016} is used.


To find an optimal point in time we used the procedure described in section \ref{sec:find-optimal-time}.  

For the nearby search, we use a list of selected OSM tags like \verb|shop=supermarket| or \verb|amenity=pharmacy| as search criteria. Only locations which have opening hours specified (via the \verb|opening_hours| tag) and are within a defined radius around the starting point are considered. We use the direct distance (“as the crow flies”). To reduce the computation effort a maximum number of results $k$ can be specified.

The implementation of Brent's method in the Apache Commons Mathematics Library \parencite{ASF2016} is used as optimization algorithm with 10 random start points to reduce the risk that only a local optimum is found.

\subsection{Results}

\subsubsection{Routing}

\begin{table}
	\centering
	\begin{tabular}{lp{9.25cm}lcc}
		\toprule
		& & \emph{temperature} & \emph{heatindex} \\
		\midrule
		\multicolumn{4}{l}{Reduction of heat exposure (\% of cases) }   \\
		& overall  & \SI{79.70}{\percent} & \SI{80.53}{\percent}  \\
		& more than \SI{5}{\percent} & \SI{42.72}{\percent} & \SI{45.11}{\percent} \\
		& more than \SI{10}{\percent} & \SI{13.81}{\percent} & \SI{16.07}{\percent} \\
		\multicolumn{4}{l}{Reduction of heat exposure}  \\
		& average  & \SI{4.63}{\percent} & \SI{4.69}{\percent}  \\
		& maximum  & \SI{25.97}{\percent} & \SI{26.17 }{\percent}  \\
		\multicolumn{4}{l}{Increase of distance}  \\
		& average & \SI{5.59}{\percent} & \SI{5.76}{\percent}  \\
		\multicolumn{4}{l}{Reduction of relative heat exposure ($w_h / w_d$)}  \\
		& average  & \SI{2.12}{\celsius} & \SI{2.32}{\celsius}  \\
		\bottomrule
	\end{tabular}
	\caption{Overview of the routing results. The values are relative to the shortest route. \label{tab:results-routing}}
\end{table}

\afterpage{
\begin{figure}
	\centering
	\subfigure[map]{
		\label{fig:route-example:map}
		\includegraphics[scale=0.7]{figures/route_example_map}
	}    
	\subfigure[thermal scan (evening)]{
		\label{fig:route-example:raster}
		\includegraphics[scale=0.7]{figures/route_example_raster}
	}     
	\caption{Routing example: both the \emph{temperature} and the \emph{heatindex} weighting found the same route. (Map tiles by \textcite{Stamen2017}, under CC BY 3.0\protect\footnotemark{}. Map data by \textcite{OSMF2016}, under ODbL\protect\footnotemark{})}
	\label{fig:route-example}
\end{figure}
\addtocounter{footnote}{-1}
\footnotetext{\url{http://creativecommons.org/licenses/by/3.0}}
\stepcounter{footnote}
\footnotetext{\url{http://www.openstreetmap.org/copyright}}
}

To evaluate the routing, we select 1000 random pairs of start and destination points from the examined  area and 10 random dates from the period of 1 June to 31 August 2015. For each of the start destination pairs and each date we  perform the evaluation at 7:00, 11:00, 15:00, 19:00 and 23:00, so overall we have $50\,000$ samples. As a benchmark, we compute the shortest path for each sample. 


An overview of our results is given in table \ref{tab:results-routing}. In many cases the heat exposure can be reduced. On average the heat exposure is decreased by \SI{\sim 4.7}{\percent} while in the same time the distance increased by at most \SI{5.76}{\percent} on average. In some cases, the heat exposure is reduced by up to \SI{25}{\percent}. The weighted average of the thermal comfort measure can be reduced by \SI{\sim 2}{\celsius} on average. There are only slight differences between the air temperature and the heat index as measure for thermal comfort. 

In the example given in figure \ref{fig:route-example} the heat exposure is reduced by \SI{17.64}{\percent} (\emph{temperature}) and \SI{18.76}{\percent} (\emph{heatindex}), while at the same time the distance only increases by  \SI{0.53}{\percent}.

\subsubsection{Optimal Time}

\begin{table}
	\centering
	\begin{tabular}{lp{9.25cm}lcc}
		\toprule
		& & \emph{temperature} & \emph{heatindex} \\
		\midrule
		\multicolumn{4}{l}{Reduction of heat exposure}   \\
		& \% of cases  & \SI{68.43}{\percent} & \SI{71.08}{\percent}  \\
		\multicolumn{4}{l}{Reduction of heat exposure}  \\
		& average  & \SI{8.09}{\percent} & \SI{7.73}{\percent}  \\
		& maximum  & \SI{62.29}{\percent} & \SI{62.88}{\percent}  \\
		\multicolumn{4}{l}{Increase of distance}  \\
		& average  & \SI{4.60}{\percent}  & \SI{4.72}{\percent}  \\
		\bottomrule
	\end{tabular}
	\caption{Overview of the results of the combined approach each compared with the reference solution.  \label{tab:results-optimal-time}}
\end{table}

\begin{figure}
	\centering
	\includegraphics[scale=1]{figures/optimaltime_route_example}
	\caption{Example for nearby search: In the graphic the starting point (black dot) as well as the locations ranked best by the respective method. (Map tiles by \textcite{Stamen2017}, under CC BY 3.0. Map data by \textcite{OSMF2016}, under ODbL)}
	\label{fig:optimaltime-route-example}
\end{figure}

For the evaluation of optimal time finding procedure we selected 750 random start points. Afterwards, one of the following four search criteria is assigned to each of the start points at random: supermarket, bakery, chemist or pharmacy. For each of the start points a random start time $t_{now}$ is selected from the period of 8:00 to 20:00. The radius is set to \SI{1000}{\meter} for all start points and the maximum number of results is set to 5. Additionally, for all start points a time buffer $t_{buff}$ of 15 minutes is assumed. 

As a reference solution, we use the closest location found during the nearby search, compute the shortest path from the starting point to this location and evaluated the heat exposure at time $t_{now}$. 

The results for the combined approach are given in table \ref{tab:results-optimal-time}. Here the average reduction compared to the routing approach is significantly higher. This is expected as the heat exposure can vary strongly with the time of the day. 

In the example given in figure \ref{fig:optimaltime-route-example} the \emph{temperature} weighting selected the same pharmacy and optimal point in time (9:27) as the reference solution. Contrary the \emph{heatindex} weighting selected a different pharmacy which is \SI{476.6}{\meter} instead of  \SI{434.5}{\meter} away from the start. Additional the method found a different optimal time (19:39) and those the heat exposure is reduced by \SI{18.49}{\percent}.  
 

% !TeX encoding = utf8
% !TeX spellcheck = en_GB

\section{Conclusion}

In this work, we proposed a two-step approach to reduce the heat stress for individuals. We achieved this goal by creating a decision support system that computes heat-optimal paths to locations as well the optimal points in time to perform a desired action. We evaluated this on typical every-day activities such as grocery shopping. 
We showed that our approach reduces the heat stress in a vast majority of the cases. On average the heat stress can be reduced by \SI{\sim 4.7}{\percent} while the trade off in additional distance is also quite low (less than \SI{5.8}{\percent} on average). We achieved these results, contrary to the existing work, for relatively small distances which average over \SI{\sim  2}{\kilo\meter}.
The impact of these results is simple, but significant. One can easily compute our approach and decide for themselves the trade-off between additional distance and the heat stress reduction.
Thanks to the very small assumptions on the data set, one can apply the approach quite easily to other cities. 
But these assumptions are also the main restriction of this work. Given the rise of smart cities and (hopefully) more available data sources, one could improve the approach with more fine-detailed data. Even more interesting would be the inclusion of intra urban temperature forecasts. By incorporating exact forecasts of future values along possible pathways, the optimal point in time as well as the reduction of the heat stress could be improved. Additionally, the inclusion of more complex heat indices could increase the validity for any potential user. 
Finally, the computation of an overall route which covers a multitude of potential points of interests would be an interesting extension. This could be used for tourists or even worker scheduling. Here not only the time and the route with minimal heat exposure should be considered but also, the ordering of the locations.  
% !TeX encoding = utf8
% !TeX spellcheck = en_GB

\section{Story}
\begin{itemize}
	\item Motivation
	\begin{itemize}
		\item heat is critical for humans
		\item particular problems for risk groups (illness, old, etc)
		\item walking to areas often done action in particular to typical locations (shopping, health care, etc)
		\item those actions have to be done   
	\end{itemize}
	\item Problem definition
	\begin{itemize}
		\item how to minimize the impact of heat stress on walking paths?
		\item subdivided into:
		\begin{itemize}
			\item for a given route?
			\item When and how to walk typical routes, e.g. pharmacy, doctor
		\end{itemize} 
	\end{itemize}
	\item method:
	\begin{itemize}
		\item Goal of this paper: We want to provide a routing method to minimize the heat stress of typical walking actions
		\item We do so by providing a value function for heat stress which can be used in route planning
	\end{itemize}
	\item Evaluation:
	\begin{itemize}
		\item Our data set
		\item Evaluation 1 on routes given date in time.  (pure routing approach) and its metrics
		\item Evaluation 2: producing queries for typical walking tasks (our selection) and their evaluation
		\item implementation as a demonstrator is produced and available at $homepage$
	\end{itemize}
	\item Conclusion and contribution
\end{itemize}


 


\printbibliography

\end{document}
