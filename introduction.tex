% !TeX encoding = utf8
% !TeX spellcheck = en_GB

\section{Introduction}

Heat is an important factor to human health and comfort. High temperatures cannot only lead to a discomfort,such as sweating, it also has serious negative effects on the health as well as on the ability to work. 

In numerous studies an increase in both mortality and
morbidity has been associated with a high ambient temperature \parencite{Basu2009}. The most well known example in recent history is the 2003 heat wave in Europe. In France alone 19490 heat related deaths were to be mourned, an excess
mortality of 60\% for the whole country. In Paris the excess mortality reached 
142\% \parencite{Trobine2008death}. 

Certain groups are especially vulnerable to heat stress such as older people or people with health problems \parencite{Ebi2004,Huebler2007}. For patients with multiple sclerosis an increased body temperature can lead to a worsening of their symptoms \parencite{Davis2010}.

Developments like the ageing of society, the increasing urbanisation and the climate change is making the adaptation to heat stress danger more and more important. Due to the tendency that a rising number of people is moving into the cities \parencite{department2014world}, the urban heat island effect (UHI) is gaining more importance in the future. The UHI effect states that an urban area can be  \SIrange{8}{12}{\celsius} warmer than the rural areas \parencite{Prashad2014}. This increased temperature can be caused by several facts. One example is that urban materials such as asphalt, concrete, and bricks are storing the energy from the sun and releasing it later to their surrounding \parencite{Prashad2014}. 

There are several possiblities that can been taken to reduce the heat stress. These range from urban planning measures to the implementation of heat warning systems \parencite{Ebi2004}. But these steps are only available on large scale projects. Individuals can reduce their heat stress by adapting their everyday behaviour. In this work we present a two step approach to help individuals in doing so.

In a city, most typical activities are in walking distance. These can range from going to a grocery store to the visit of a doctor. And while these activities can not be omitted, it is possible to use different routes or change the time when they are conducted. In doing so, one can easily reduce the heat stress without any negative impact on the quality of life. 

In this paper we use this reasoning into a two step approach to help individuals reduce their heat stress. First, we apply a time dependent routing algorithm to compute the optimal path in regard to the heat stress. This algorithm is then used to determine the optimal point in time to conduct typical everyday activities. 



   
\subsection{Related Work} 

\subsubsection{Heat Stress}
The impact of heat on the human body has long been a subject of study. In particular , the term thermal comfort plays a key role, which describes climatic conditions consider comfortable, i.e. neither to warm nor to cold.

Heat can be defined in several ways, where the most simple one is the air temperature. But other, more in depth approaches consider additional factors such as e.g. humidity and physical activity \parencite{Staiger2011,Huebler2007}. Staiger et al.\parencite{Staiger1997} state that only a complete heat budget model of the human body is sufficient to make any reliable statements regarding the influence of heat on the body. They present such a model and update this model to include factors such as sweating in a follow up paper\parencite{Staiger2011}. 

As such exhaustive models are quite difficult to compute and the necessary data is not always available, heat indices are often employed. The most well known indices that consider a complete human heat budget model are the following:
\begin{inparaenum}[(1)]
  \item \citeauthor{Steadman1979}'s heat index (\citeauthor{Steadman1979} \citeyear*{Steadman1979}\citeyear*{Steadman1979a}),
  \item the predicted mean vote (PMV) \parencite{Fanger1973},
  \item the perceived temperature \parencite{Staiger1997,Jendritzky2000},
  \item and the universal thermal climate Index UTCI \parencite{Jendritzky2010}.
\end{inparaenum}

For all these indices the following meteorological parameters are important:
\begin{inparaenum}[(1)]
\item air temperature,
\item vapour pressure,
\item wind velocity 
\item and mean radiant temperature of the surroundings \parencite{Matzarakis1999}.
\end{inparaenum}

Based on the availability of data, in this paper we will use \citeauthor{Steadman1979}'s heat index \parencite{Steadman1979} and, as a simple comparison measure, the air temperature. To compute an approximation of the heat index we use the formula published by \textcite[77]{Stull2011}.  

\subsubsection{Time Dependent Routing}
Several research projects have considered environmental factors for pedestrian routing in the paste. The AffectRoute routing algorithm proposed by \textcite{Huang2014} for instance takes the affective responses to the environment into account, e.g. to find a route that a person considers safer.  \textcite{Sharker2012} are proposing a method to find a health optimal route, considering several environmental factors like complexity of the walking trail (slope etc.) and weather (only “Good”, “Fair” or “Bad”). A method to find a route with a minimal pollution exposure has been proposed by \textcite{Hasenfratz2015} in his PhD thesis.

The NaviComf framework for pedestrian routing proposed by \citeauthor{Dang2012} (\citeyear{Dang2012}, \citeyear{Dang2013}), enables to improve the comfort considering environmental factors varying over time. The proposed framework uses a multi-factor cost model for the evaluation of the route and enables them to consider heterogeneous environmental information from multi-modal
 sensors like air temperature and humidity. To find a optimal route \textcite{Dang2013} are proposing three different algorithms,  a bounded depth-first search algorithm, an adjustable dynamic planning algorithm and a heuristic particle planning algorithm. As a sample application, the authors implemented a routing app for thermal comfort navigation. The meteorological used for this sample application have been collected using a network of 40 micro-climate sensor nodes which detected air temperature and relative humidity. 

In our approach we 

%TODO anpassen
To find a route with minimal heat exposure we are using a different approach then \citeauthor{Dang2012} (\citeyear{Dang2012}, \citeyear{Dang2013}). First of all we didn't have data of a mobile sensor network at hand, instead we're used the remote sensing data of thermal scanner flight as well as the data of weather station. Another contribution is the comparison of different thermal comfort measures like air temperature and heat index. A further difference is the application of static routing algorithm instead of dynamic one as used by \citeauthor{Dang2012}.
